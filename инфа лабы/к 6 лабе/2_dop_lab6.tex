\documentclass[12pt, a4paper, twoside]{article}
%\usepackage[a4paper, total={6in, 10in}]{geometry}
\usepackage[english, russian]{babel}
\usepackage[a4paper,left=5cm,right=5cm,top=2cm,bottom=1cm,footskip=.5cm]{geometry}

\usepackage[utf8]{inputenc}
\usepackage{lipsum}
\usepackage{blindtext}
\pagestyle{empty}

\begin{document}
\bf
74\\
\begin{center}
НАЧАЛА ЕВКЛИДА
\end{center}
сходятся; значит, \textit{EB}, \textit{ID}, продолжение в сторону \textit{B} и \textit{D}, сойдутся. Продолжим их и пусть они сойдутся в \textit{H}; соединим \textit{AH}. И поскольку \textit{AC} равна \textit{CE}, и угол \textit{EAC} равен \textit{AEC} (предложение 5 книги 1); и угол при \textit{C} прямой; значит, каждый из углов \textit{EAC}, \textit{AEC} - половина прямого
(предложение 32 книги 1). Вследствие того же вот и
каждый из углов \textit{СЕВ}, \textit{ЕВС} - половина прямого; значит,
угол \textit{АЕВ} прямой. И поскольку угол \textit{ЕВС}-половина
прямого, то значит, и угол \textit{DBH} половина прямого (пред-
ложение 15 книги 1). Но и угол \textit{BDH} прямой, ибо он
равен углу \textit{DCE}; они накрестлежащие (предложение 29
книги I); значит, остающийся угол \textit{DHB} половина прямого;
значит, угол \textit{DHB} равен \textit{DBH}, так что и сторона \textit{BD}
равна стороне \textit{HD} (предложение 6 книги 1). Далее, но-
скольку угол \textit{EHI} - половина прямого, угол же при
прямой, ибо он равен противоположному углу при \textit{С}
(предложение 34 книги I); значит, остающийся угол \textit{IEH}-
половина прямого (предложение 32 книги 1); значит, угол
\textit{EHI} равен \textit{IEH}, так что и сторона \textit{HI} равна стороне \textit{EI}.
И поскольку [\textit{ЕС} равна \textit{СА} и] квадрат на \textit{ЕС} равен квад-
рату на \textit{СА}; значит, квадраты на \textit{ЕС} и \textit{СА} (вместе) вдвое
больше квадрата на \textit{СА}. Квадратам же на \textit{ЕС} и \textit{СА} равен
квадрат на \textit{EA} (предложение 47 книги I); значит, квадрат
на \textit{ЕА} вдвое больше квадрата на \textit{АС}. Далее, поскольку
\textit{IH} равна \textit{EI}, и квадрат на \textit{IН} равен квадрату на \textit{IE}, значит, квадраты на \textit{HI} и на \textit{IE} <вместе> вдвое больше квадрата на \textit{EI}. Квадратам же на \textit{HI} и \textit{IE} <вместе> равен
квадрат на \textit{ЕН} (предложение 47 книги I); значит, квадрат
на \textit{EH} вдвое больше квадрата на \textit{EI}. Но \textit{EI} равна \textit{CD}
(предложение 34 книги I), значит квадрат на \textit{ЕН} вдвое
больше квадрата на \textit{CD}. Доказано же, что и квадрат на
\textit{EА} вдвое больше квадрата на \textit{АС}, значит, квадраты на \textit{АE}
и \textit{ЕН} <вместе> вдвое больше <вместе взятых> квадратов
на \textit{АС} и \textit{CD}. Квадратам же на \textit{АЕ} и \textit{EН} равен квадрат
на \textit{АН} (предложение 47 книги I); значит, квадрат на \textit{АН}
вдвое больше квадратов на \textit{АС} и \textit{CD}. Квадрату же на \textit{АН}
равны <вместе взятые> квадраты на \textit{AD} и \textit{DH}; значит,
квадраты на \textit{AD} и на \textit{DH} вдвое больше квадратов
на \textit{AC} и \textit{CD}. Но \textit{DH} равна \textit{DB}; значит, квадраты

\end{document}